% !TeX root = ../thesis.tex


\chapter*{Kurzfassung}
\addcontentsline{toc}{chapter}{Kurzfassung}

Die vorliegende Masterarbeit besch\"aftigt sich mit der Entwicklung eines Ansatzes zur Generierung von kooperativen Fahrstreifenwechseltrajektorien f\"ur hochautomatisierte Fahrzeuge.
Bei dem Ansatz wird auf eine explizite Kommunikation zwischen den Fahrzeugen verzichtet.
Dadurch erlaubt der Ansatz ein kooperatives Verhalten zwischen dem automatisierten Fahrzeug und den von Menschen gef\"uhrten Fahrzeugen.

In dem entwickelten Ansatz findet eine Trajektorieplanung f\"ur einen Fahrsteifenwechsel statt, bei der von einem kooperativen Verhalten der beteiligten Fahrzeuge ausgegangen wird.
Diese Annahme erm\"oglicht, dass auch bei dichtem Verkehr eine valide Fahrstreifenwechseltrajektorie gefunden werden kann.
Die Ermittlung der Fahrstreifenwechseltrajektorie findet in einem Optimierungsverfahren statt.
Um die Interaktion mit den Fahrzeugen in der Umgebung des automatisierten Fahrzeuges zu modellieren wird eine Multi-Agenten-Optimierung angewandt. 
Dabei wird die Bewegungspr\"adiktion der anderen Fahrzeuge in das Optimierungsproblem integriert.
Das kooperative Verhalten wird durch ein Gesamtkostenfunktional modelliert, bei dem zus\"atzlich zu den Kosten des Ego"=Fahrzeuges auch die Kosten der am Fahrstreifenwechselman\"over beteiligten Fahrzeuge ber\"ucksichtigt werden.

Das Optimierungsproblem wird durch einen samplingbasierten Ansatz gel\"ost.
Dazu werden zuf\"allige Geschwindigkeitsprofile erzeugt.
F\"ur jedes der Geschwindigkeitsprofile wird ein Fahrstreifenwechselpfad erzeugt.
Anschlie{\ss}end wird f\"ur die sich daraus ergebende Trajektorie bestimmt, welche Fahrzeuge auf dem Zielfahrstreifen direkt an dem Fahrstreifenwechselman\"over beteiligt sind.
F\"ur diese Fahrzeuge wird in einer inneren Optimierung eine kooperative Trajektorie pr\"adiziert und so ein Trajektorienset mit einem kooperativen Fahrstreifenwechsel erzeugt.
Aus allen erzeugten Trajektoriensets wird dasjenige mit den geringsten Gesamtkosten ausgew\"ahlt.

Der entwickelte Ansatz wird simulativ validiert. 
Zuerst werden die Ergebnisse eines einzelnen Planungsschrittes in verschieden Szenarien analysiert. 
Anschlie{\ss}end wird untersucht wie sich das automatisierte Fahrzeug bei kooperativen und unkooperativen Verhalten der Fahrzeuge auf dem Zielfahrstreifen verh\"alt, unter Ber\"ucksichtigung fortlaufender Neuplanung der Trajektorien.

\textbf{Schlagw\"orter}: Trajektorienplanung, kooperative Verhaltensplanung, Multi"=Agenten"=Optimimierung, Fahrstreifenwechsel, Kostenfunktional, Pfadplanung.



\chapter*{Abstract}
\addcontentsline{toc}{chapter}{Abstract}
In this master thesis an approach for generation of cooperative lane change trajectories for highly automated vehicles is developed.
The approach doesn't make use of an explicit vehicle-to-vehicle communication.
Therefor it is suitable for cooperation between the automated vehicle and human-driven vehicles.

The trajectory planning algorithm of this approach assumes a cooperative behavior of all involved vehicles.
Therefor a lane change can take place even in dense traffic.
The lane changing trajectory is generated in an optimization.
To model the interaction between the ego vehicle and vehicles in its surrounding a multi-agent-optimization is used.
Thus predictive paths of the other vehicles are integrated into the optimization problem.
In order to model the cooperative behavior of the involved vehicles, a new cost functional is introduced, containing not only the cost of the ego vehicle but also the cost of all vehicles involved in the lane change maneuver.

The optimization problem is solved by a sampling based approach.
For this purpose random velocity profiles are generated.
For each of these velocity profiles, a path is generated resulting in a trajectory for each of the velocity profiles.
Subsequently for each trajectory, it is determined which of the vehicles are directly involved in the lane changing maneuver.
For each of these vehicles an cooperative trajectory is predicted in an inner optimization process resulting in a trajectory set of a cooperative lane change maneuvers.
The generated trajectory sets are evaluated by the presented cost functional. 
The trajectory set resulting in the lowest cost is chosen as the solution of the optimization problem.

The presented approach is evaluated in simulation.
At first the results of a single planning step are analyzed.
Subsequently, an evaluation of the automated vehicle's behavior when dealing with cooperative and uncooperative vehicles on the traget lane is done.
This is achieved by consideration of continuous replanning of the trajectory.



\textbf{Keywords}: Trajectory planning, cooperative behavior planning, multi agent optimization, lane change maneuver, cost functional, path planning.